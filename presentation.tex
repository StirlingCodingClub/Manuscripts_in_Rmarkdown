\documentclass[ignorenonframetext,]{beamer}
\setbeamertemplate{caption}[numbered]
\setbeamertemplate{caption label separator}{: }
\setbeamercolor{caption name}{fg=normal text.fg}
\beamertemplatenavigationsymbolsempty
\usepackage{lmodern}
\usepackage{amssymb,amsmath}
\usepackage{ifxetex,ifluatex}
\usepackage{fixltx2e} % provides \textsubscript
\ifnum 0\ifxetex 1\fi\ifluatex 1\fi=0 % if pdftex
\usepackage[T1]{fontenc}
\usepackage[utf8]{inputenc}
\else % if luatex or xelatex
\ifxetex
\usepackage{mathspec}
\else
\usepackage{fontspec}
\fi
\defaultfontfeatures{Ligatures=TeX,Scale=MatchLowercase}
\fi
% use upquote if available, for straight quotes in verbatim environments
\IfFileExists{upquote.sty}{\usepackage{upquote}}{}
% use microtype if available
\IfFileExists{microtype.sty}{%
\usepackage{microtype}
\UseMicrotypeSet[protrusion]{basicmath} % disable protrusion for tt fonts
}{}
\newif\ifbibliography
\usepackage{longtable,booktabs}
\usepackage{caption}
% These lines are needed to make table captions work with longtable:
\makeatletter
\def\fnum@table{\tablename~\thetable}
\makeatother
\usepackage{graphicx,grffile}
\makeatletter
\def\maxwidth{\ifdim\Gin@nat@width>\linewidth\linewidth\else\Gin@nat@width\fi}
\def\maxheight{\ifdim\Gin@nat@height>\textheight0.8\textheight\else\Gin@nat@height\fi}
\makeatother
% Scale images if necessary, so that they will not overflow the page
% margins by default, and it is still possible to overwrite the defaults
% using explicit options in \includegraphics[width, height, ...]{}
\setkeys{Gin}{width=\maxwidth,height=\maxheight,keepaspectratio}

% Prevent slide breaks in the middle of a paragraph:
\widowpenalties 1 10000
\raggedbottom

\AtBeginPart{
\let\insertpartnumber\relax
\let\partname\relax
\frame{\partpage}
}
\AtBeginSection{
\ifbibliography
\else
\let\insertsectionnumber\relax
\let\sectionname\relax
\frame{\sectionpage}
\fi
}
\AtBeginSubsection{
\let\insertsubsectionnumber\relax
\let\subsectionname\relax
\frame{\subsectionpage}
}

\setlength{\parindent}{0pt}
\setlength{\parskip}{6pt plus 2pt minus 1pt}
\setlength{\emergencystretch}{3em}  % prevent overfull lines
\providecommand{\tightlist}{%
\setlength{\itemsep}{0pt}\setlength{\parskip}{0pt}}
\setcounter{secnumdepth}{0}

\title{Manuscripts in Rmarkdown}
\author{Brad Duthie}
\date{31 October 2018}

\begin{document}
\frame{\titlepage}

\begin{frame}

\end{frame}

\begin{frame}{Stirling Coding Club (SCC) on GitHub}

\textbf{A lot of learning \& discussion in SCC occurs on line in
\href{http://github.com}{GitHub}.}

\begin{figure}[htbp]
\centering
\includegraphics{images/github_logo.png}
\caption{}
\end{figure}

\begin{itemize}
\tightlist
\item
  \href{https://stirlingcodingclub.github.io/Manuscripts_in_Rmarkdown/Rmarkdown_notes.html}{Full
  notes} are available online
\item
  \href{https://github.com/StirlingCodingClub/Manuscripts_in_Rmarkdown}{Repository}
  to see all today's code
\item
  \href{https://github.com/StirlingCodingClub/Manuscripts_in_Rmarkdown/issues}{Issues}
  online to ask questions
\end{itemize}

\textbf{If you are not yet signed up, go to
\url{http://github.com/join}, then contact
\href{https://bradduthie.github.io/}{Brad} or
\href{https://www.stir.ac.uk/people/266756}{Anna}.}

\end{frame}

\begin{frame}{Stirling Coding Club (SCC) on GitHub}

\textbf{This slide presentation is available in the link below}

\url{https://stirlingcodingclub.github.io/Manuscripts_in_Rmarkdown/presentation.html}

\begin{itemize}
\tightlist
\item
  \textbf{Students and educators can sign up for a
  \href{https://education.github.com/pack}{GitHub Education} package,
  which includes a lot of free coding software}
\item
  \href{http://github.com}{GitHub} unlimited free private repositories
  (usually 7 USD per month)
\item
  \href{http://gitkraken.com}{GitKraken} free pro account (usually 49
  USD per year)
\end{itemize}

\begin{figure}[htbp]
\centering
\includegraphics{images/gitkraken.png}
\caption{}
\end{figure}

\end{frame}

\begin{frame}{Edinburgh Coding Club Invitation}

Stirling Coding Club participants are invited to take part in upcoming
\href{https://ourcodingclub.github.io/}{workshops at the University of
Edinburgh}.

\begin{itemize}
\tightlist
\item
  \textbf{7 NOV 2018: 15:00-17:00}. Introduction to mixed effects
  models.
\item
  \textbf{21 NOV 2018: 15:00-17:00}. Bayesian meta-analysis (uses the
  MCMCglmm package)
\item
  \textbf{5 DEC 2018: 15:00-17:00}. Introduction to
  \href{https://inla.r-inla-download.org/r-inla.org/papers/inla-rss.pdf}{Integrated
  Nested Laplace Approximations} (INLA) with R (spatial autocorrelation
  and mixed models).
\end{itemize}

\textbf{\url{https://ourcodingclub.github.io}}

\end{frame}

\begin{frame}{Objectives for learning Rmarkdown}

\begin{enumerate}
\def\labelenumi{\arabic{enumi}.}
\tightlist
\item
  \textbf{Understand the features of Rmarkdown and why using it to write
  scientific documents may be useful}
\item
  \textbf{Create an Rmarkdown file and assemble it into an HTML, PDF, or
  DOCX document using knitr in Rstudio}
\item
  \textbf{Apply basic integration of R code into Rmarkdown to analyse
  data and plot results in output}
\item
  \textbf{Be able to navigate to the
  \href{https://stirlingcodingclub.github.io/Manuscripts_in_Rmarkdown/Rmarkdown_notes.html}{accompanying
  Rmarkdown notes} and make use of them for additional tools}
\item
  \textbf{Continue asking questions and sharing tips in the Rmarkdown
  repository
  \href{https://github.com/StirlingCodingClub/Manuscripts_in_Rmarkdown/issues}{issues
  page} on GitHub}
\end{enumerate}

\end{frame}

\begin{frame}{Where did Rmarkdown come from?}

\begin{longtable}[]{@{}ll@{}}
\toprule
Microsoft Word (1983) & \(\LaTeX{}\) (1980)\tabularnewline
\midrule
\endhead
- Used in the life sciences & - Used in maths and physics\tabularnewline
- What you see is what you get & - Edit files in
\href{https://en.wikipedia.org/wiki/LaTeX}{plain text}
(code)\tabularnewline
- Proprietary software & - Free software\tabularnewline
- \textbf{Low learning curve} & - \textbf{High learning
curve}\tabularnewline
- \textbf{No analysis integration} & - \textbf{No analysis
integration}\tabularnewline
\bottomrule
\end{longtable}

\textbf{Rmarkdown} (2012) is free software with a \textbf{relatively low
learning curve} in which authors write in plain text and can easily
integrate R analyses, citations, and tables or figures.

\end{frame}

\begin{frame}{Why is Rmarkdown worth learning?}

\begin{itemize}
\tightlist
\item
  Learning is a relatively low additional time investment if already
  invested in R
\item
  Produces high quality
  \href{https://stirlingcodingclub.github.io/Manuscripts_in_Rmarkdown/ms_history/ms_final.html}{HTML},
  \href{https://github.com/StirlingCodingClub/Manuscripts_in_Rmarkdown/blob/224e0f3673aece576d5c859f5409b6c9b68a5565/ms.pdf}{PDF},
  and
  \href{https://github.com/StirlingCodingClub/Manuscripts_in_Rmarkdown/raw/224e0f3673aece576d5c859f5409b6c9b68a5565/ms.docx}{DOCX}
  documents with the push of a button from an
  \href{https://github.com/StirlingCodingClub/Manuscripts_in_Rmarkdown/blob/master/ms.Rmd}{Rmd
  file} in Rstudio
\item
  Removes the need to format citations manually (with BibTeX)
\item
  Allows users to insert images and equations seamlessly
\item
  \textbf{Complete integration of data analysis and manuscript} (no
  copy-pasting when values or figures change)
\end{itemize}

\textbf{You do not need to learn everything at once for Rmarkdown to be
useful. If you get stuck or cannot figure out how to do something, you
can always knit a DOCX and work from there.}

\end{frame}

\end{document}
